\documentclass[handout]{beamer}
% Class options include: notes, notesonly, handout, trans,
%                        hidesubsections, shadesubsections,
%                        inrow, blue, red, grey, brown
\mode<presentation>
{
  \usetheme{CambridgeUS}
  % or ...
  \usecolortheme{beaver}
  \setbeamercovered{transparent}
  % or whatever (possibly just delete it)
}
\setbeamertemplate{navigation symbols}{}


\usepackage[english]{babel}
% or whatever

\usepackage[utf8]{inputenc}
% or whatever

\usepackage{times}
\usepackage[T1]{fontenc}
% Or whatever. Note that the encoding and the font should match. If T1
% does not look nice, try deleting the line with the fontenc.
\usepackage[absolute,overlay]{textpos}


\title[People Counting and Patterns of Movement] % (optional, use only with long paper titles)
{Green IT - People Counting and Detection of Patterns of Movement}

\subtitle
{}
\author[António Calçada Novais] % (optional, use only with lots of authors)
{António Calçada Novais \\ Student nº 57432 \\ MERC}
% - Give the names in the same order as the appear in the paper.
% - Use the \inst{?} command only if the authors have different
%   affiliation.

\institute[IST] % (optional, but mostly needed)
{
  INESC-ID, Communications Networks and Mobility\\
  Instituto Superior Técnico - Campus TagusPark
}
% - Use the \inst command only if there are several affiliations.
% - Keep it simple, no one is interested in your street address.

\date % (optional, should be abbreviation of conference name)
{November 11, 2011}
% - Either use conference name or its abbreviation.
% - Not really informative to the audience, more for people (including
%   yourself) who are reading the slides online

\subject{Computer Science}
% This is only inserted into the PDF information catalog. Can be left
% out. 



% If you have a file called "university-logo-filename.xxx", where xxx
%LOGOS
\pgfdeclareimage[height=0.6cm]{logoIST}{figures/logoIST.png}
\pgfdeclareimage[height=0.8cm]{logoINESC}{figures/inescID.png}
\pgfdeclareimage[height=0.3cm]{logoMIT}{figures/mit-logo.png}

%ARCHITECTURE
\pgfdeclareimage[height=4.9cm]{topoFIS}{figures/novais_arch2.pdf}
\pgfdeclareimage[height=4.5cm]{topoLOG}{figures/novais_arch.pdf}
\pgfdeclareimage[height=5.7cm]{arch_top}{figures/top_level_arch.pdf}
\pgfdeclareimage[height=5cm]{arch_gsd_plat}{figures/gsd_platform.pdf}
\pgfdeclareimage[height=4.5cm]{arch_spotter}{figures/spotter_v2.pdf}

%MAP PLANT/CELL
\pgfdeclareimage[height=5cm]{cells11-5}{figures/cells11-5.pdf}
%\pgfdeclareimage[height=5cm]{planta11-5}{figures/planta11-5.pdf}
\pgfdeclareimage[height=4cm]{planta11}{figures/planta11.pdf}

%PROTOTYPES
\pgfdeclareimage[height=5cm]{ts7500}{figures/ts7500.pdf}
\pgfdeclareimage[height=3.2cm]{ir_proto}{figures/ir_proto.jpg}
\pgfdeclareimage[height=3.2cm]{ir_emitters}{figures/ir_emitters.jpg}
%\pgfdeclareimage[height=3.2cm]{ir_entrance1}{figures/ir_entrance1.jpg}
%\pgfdeclareimage[height=3.2cm]{ir_entrance2}{figures/ir_entrance2.jpg}
\pgfdeclareimage[height=3.2cm]{ir_receivers}{figures/ir_receivers.jpg}
\pgfdeclareimage[height=3.2cm]{bt_proto}{figures/bt_simple.jpg}

%TESTS
\pgfdeclareimage[height=4cm]{test_mov}{figures/movimento.pdf}
\pgfdeclareimage[height=3.6cm]{test1.2}{figures/13-16zona2.pdf}
\pgfdeclareimage[height=3.6cm]{test1.3}{figures/13-16zona3.pdf}
\pgfdeclareimage[height=3.6cm]{test2.2}{figures/17-20zona2.pdf}
\pgfdeclareimage[height=3.6cm]{test2.3}{figures/17-20zona3.pdf}
\pgfdeclareimage[height=3.6cm]{test3.2}{figures/21-23zona2.pdf}
\pgfdeclareimage[height=3.6cm]{test3.3}{figures/21-23zona3.pdf}
% is a graphic format that can be processed by latex or pdflatex,
% resp., then you can add a logo as follows:


\setlength{\TPHorizModule}{1mm}
\setlength{\TPVertModule}{1mm}
\newcommand{\MyLogo}{%
\begin{textblock}{14}(105.0,5.0)
  \vbox{\hbox to 2.17cm{\hfil\pgfuseimage{logoIST}}\vskip-0.8cm\hbox{\pgfuseimage{logoINESC}}\vskip-0.6cm\hskip-1.4cm\hbox{\pgfuseimage{logoMIT}}}
\end{textblock}
}

%\logo{}

%\AtBeginSection[] {
%	\begin{frame}<beamer>{Índice}
%	\MyLogo
%	\tableofcontents[currentsection,subsubsectionstyle=hide]
%	\end{frame}
%}



% Theme for beamer presentation.
%\usepackage{beamerthemesplit} 
% Other themes include: beamerthemebars, beamerthemelined, 
%                       beamerthemetree, beamerthemetreebars  

%\title{Green IT - People Counting and Detection of Patterns of Movement}    % Enter your title between curly braces
%\author{António Calçada Novais}                 % Enter your name between curly braces
%\institute{NESC-ID, Communications Networks and Mobility\\Instituto Superior Técnico - Taguspark}      % Enter your institute name between curly braces
%\date{\today}                    % Enter the date or \today between curly braces

\begin{document}

% Creates title page of slide show using above information
\begin{frame}
  \MyLogo
  \titlepage
\end{frame}
\note{Talk for 20 minutes} % Add notes to yourself that will be displayed when
                           % typeset with the notes or notesonly class options

\section[Índice]{}

% Creates table of contents slide incorporating
% all \section and \subsection commands
\begin{frame}
  \MyLogo
  \frametitle{Índice}
  \tableofcontents
\end{frame}

\section{Introdução}

%\subsection{Motivação}

\begin{frame}
  \MyLogo 
  \frametitle{Introdução}   % Insert frame title between curly braces
  \framesubtitle{Motivação}
  \begin{itemize}\setlength{\itemsep}{7mm}
	\item<1-> Dificuldade de gerir consumos energéticos num campus
	\item<2-> Redução de custos energéticos é uma necessidade
	\item<3-> Previsível diminuição dos recursos naturais
	\item<4-> Disseminação da computação pervasiva 
  \end{itemize}
\end{frame}
\note[enumerate]       % Add notes to yourself that will be displayed when
{                      % typeset with the notes or notesonly class options
\item Note for Point 1   
\item Note for Point 2   
}

%\subsection{Objectivos}

\begin{frame}
  \MyLogo
  \frametitle{Introdução}   % Insert frame title between curly braces
  \framesubtitle{Objectivos}
  \begin{itemize}\setlength{\itemsep}{7mm}
	\item<1-> Desenhar um sistema capaz de contar pessoas e identificar os seus padrões de mobilidade dentro de edifícios
	\item<2-> O custo do sistema deve ser inferior às poupanças
	\item<3-> Caso de uso de plataforma multi-funcional de monitorização
	\item<4-> Integrar na plataforma supra-mencionada módulo de descoberta de serviços
  \end{itemize}
\end{frame}
\note[enumerate]       % Add notes to yourself that will be displayed when
{                      % typeset with the notes or notesonly class options
\item Note for Point 1   
\item Note for Point 2   
}

\begin{frame}
  \MyLogo
  \frametitle{Introdução}
\framesubtitle{Vantagens de uso da plataforma}   % Insert frame title between curly braces

  \begin{itemize}\setlength{\itemsep}{8mm}
			\item<1-> Transparência de detalhes de transporte
 		\item<2-> Lidar com requisitos de determinadas aplicações ao nível de rede como a sincronização
			\item<3-> Abstrair de dificuldades que surjam na comunicação
			\item<4-> Possibilidade de inserir novos módulos que lidem com outro tipo de requisitos como compressão e encriptação
 %\item<2-> Optimizar o processamento à capacidade do nó
  \end{itemize}
\end{frame}
\note[enumerate]       % Add notes to yourself that will be displayed when
{                      % typeset with the notes or notesonly class options
\item Note for Point 1
\item Note for Point 2
}

%\section{State of the art}

%\begin{frame}
%  \frametitle{State of the art}   % Insert frame title between curly braces

%  \begin{itemize}
%  \item Point 1
%  \item Point 2
%  \item Point 3
%  \end{itemize}
%\end{frame}
%\note[enumerate]       % Add notes to yourself that will be displayed when
%{                      % typeset with the notes or notesonly class options
%\item Note for Point 1   
%\item Note for Point 2   
%}%

%\subsection{Service Discovery}

%\begin{frame}
%  \frametitle{Simple slide with three points shown in succession}   % Insert frame title between curly braces%
%
 % \begin{itemize}
  %\item<1-> Point 1 (Click ``Next Page'' to see Point 2) % Use Next Page to go to Point 2
  %\item<2-> Point 2  % Use Next Page to go to Point 3
  %\item<3-> Point 3
  %\end{itemize}
%\end{frame}
%\note{Speak clearly}  % Add notes to yourself that will be displayed when
                      % typeset with the notes or notesonly class options

%\subsection{Location Systems}

%\begin{frame}
%  \frametitle{Simple slide with three points shown in succession}   % Insert frame title between curly braces%

%  \begin{itemize}
%  \item<1-> Point 1 (Click ``Next Page'' to see Point 2) % Use Next Page to go to Point 2
 % \item<2-> Point 2  % Use Next Page to go to Point 3
  %\item<3-> Point 3
  %\end{itemize}
%\end{frame}
%\note{Speak clearly}  % Add notes to yourself that will be displayed when
                      % typeset with the notes or notesonly class options



\section{Arquitectura}

\subsection{Sistema de localização}

\begin{frame}
  \MyLogo
  \frametitle{Sistema de localização}\framesubtitle{Requisitos}   % Insert frame title between curly braces

  \begin{itemize}\setlength{\itemsep}{6mm}
  \item<1-> Não intrusivo %(Click ``Next Page'' to see Point 2) % Use Next Page to go to Point 2
  \item<2-> Escalável a diferentes edifícios e espaços
  \item<3-> Flexivel quanto às tecnologias  % Use Next Page to go to Point 3
  \item<4-> Baixa complexidade de computação
  \item<5-> Manter privacidade das pessoas
  \end{itemize}
\end{frame}
\note{Speak clearly}  % Add notes to yourself that will be displayed when
                      % typeset with the notes or notesonly class options

\begin{frame}
  \MyLogo
  \frametitle{Sistema de localização}\framesubtitle{Topologia Física}   % Insert frame title between curly braces
  \begin{columns}
   \column{.5\textwidth}
       \pgfuseimage{topoFIS}
   
   \column{.5\textwidth}
        \begin{itemize}\setlength{\itemsep}{3mm}
               %\item<1-> Exemplo de topologia física de rede  %(Click ``Next Page'' to see Point 2) % Use Next Page to go to Point 2
	  	\item<1-> Nós espalhados pela área de monitorização
	  	\item<2-> Nós contêm diferentes tipos de sensores  % Use Next Page to go to Point 3
	  	\item<3-> Informação é agrupada e enviada para um servidor para armazenamento
        \end{itemize}
 \end{columns}
	%\vbox{\hbox to 2.17cm{\hfil\pgfuseimage{topoFIS}}
  	%  \fbox{\begin{minipage}[t]{150pt}
    	%	%\includegraphics[height=100pt,width=150pt]{test}
	%	\pgfuseimage{topoFIS}
  	%\end{minipage}}
 %\hfill
 % \fbox{\begin{minipage}[t]{150pt}
 %   	\begin{itemize}
%	  \item<1-> Exemplo de topologia física de rede  %(Click ``Next Page'' to see Point 2) % Use Next Page to go to Point 2
%	  \item<2-> Nós espalhados pela área de monitorização
%	  \item<3-> Nós contêm diferentes tipos de sensores  % Use Next Page to go to Point 3
%	  \item<4-> Informação é agrupada e enviada para um servidor para armazenamento
%	  \end{itemize}
 % \end{minipage}}
\end{frame}
\note{Speak clearly}  % Add notes to yourself that will be displayed when
                      % typeset with the notes or notesonly class options

\begin{frame}
  \MyLogo
  \frametitle{Sistema de localização}\framesubtitle{Topologia Lógica}   % Insert frame title between curly braces
	%\vbox{\hbox to 2.17cm{\hfil\pgfuseimage{topoLOG}}
  \begin{columns}
   \column{.5\textwidth}
       \pgfuseimage{topoLOG}
   
   \column{.5\textwidth}
        \begin{itemize}\setlength{\itemsep}{2mm}
               \item<1-> Topologia Hierárquica  %(Click ``Next Page'' to see Point 2) % Use Next Page to go to Point 2
	  	\item<2-> Spotters %estão associados a zonas e interagem com sensores
		\item<3-> Zone Managers %realizam a computação de localização e contagem na zona
		\item<4-> Aggregator %é o nó sink que agrega a informação de localização de todas as zonas
		\item<5-> Servidor UMP %armazena a informação ao longo do tempo e calcula padrões de movimento
		%\item<2-> Nós espalhados pela área de monitorização
	  	%\item<3-> Nós contêm diferentes tipos de sensores  % Use Next Page to go to Point 3
	  	%\item<4-> Informação é agrupada e enviada para um servidor para armazenamento
        \end{itemize}
 \end{columns}
  	%\pgfuseimage{topoLOG}
\end{frame}
\note{Speak clearly}  % Add notes to yourself that will be displayed when
                      % typeset with the notes or notesonly class options

\subsection{Plataforma Multi-Funcional de Monitorização}

\begin{frame}
  \MyLogo
  \frametitle{Plataforma Multi-Funcional}\framesubtitle{Overview}   % Insert frame title between curly braces

  \begin{columns}
   \column{.5\textwidth}
       \pgfuseimage{arch_top}
   
   \column{.5\textwidth}
        \begin{itemize}\setlength{\itemsep}{2mm}
               %\item<1-> Exemplo de topologia física de rede  %(Click ``Next Page'' to see Point 2) % Use Next Page to go to Point 2
	  	\item<1-> Múltiplas aplicações por nó
	  	\item<2-> Spotter acede directamente ao hardware  % Use Next Page to go to Point 3
	  	\item<3-> Uso dos módulos de:
									 \begin{itemize}
               %\item<1-> Exemplo de topologia física de rede  %(Click ``Next Page'' to see Point 2) % Use Next Page to go to Point 2
	  	\item Sincronização
	  	\item Transporte Não-fiável  % Use Next Page to go to Point 3
			 \item Descoberta de Serviços
	  	\item Configuração
	
        \end{itemize}
        \end{itemize}
 \end{columns}
\end{frame}
\note{Speak clearly}  % Add notes to yourself that will be displayed when
                      % typeset with the notes or notesonly class options

\begin{frame}
  \MyLogo
  \frametitle{Plataforma Multi-Funcional}\framesubtitle{Descoberta de Serviços}   % Insert frame title between curly braces

  \begin{columns}
   \column{.5\textwidth}
       \pgfuseimage{arch_gsd_plat}
   
   \column{.5\textwidth}
        \begin{itemize}\setlength{\itemsep}{6mm}
               %\item<1-> Exemplo de topologia física de rede  %(Click ``Next Page'' to see Point 2) % Use Next Page to go to Point 2
	  	\item<1-> Funciona como um add-on à plataforma
	  	\item<2-> Fornece uma API para que outras aplicações possam usar o módulo  % Use Next Page to go to Point 3
	  	\item<3-> Funcionamento baseado no protocolo GSD
        \end{itemize}
   \end{columns}
\end{frame}
\note{Speak clearly}  % Add notes to yourself that will be displayed when
                      % typeset with the notes or notesonly class options

\subsection{Componentes de localização}

\begin{frame}
  \MyLogo
  \frametitle{Componentes de localização}\framesubtitle{Spotter}   % Insert frame title between curly braces
  \begin{columns}
   \column{.5\textwidth}
       \pgfuseimage{arch_spotter}
   
   \column{.5\textwidth}
        \begin{itemize}\setlength{\itemsep}{6mm}
               %\item<1-> Exemplo de topologia física de rede  %(Click ``Next Page'' to see Point 2) % Use Next Page to go to Point 2
	  	\item<1-> Sistema de plugins
	  	\item<2-> Síncronos ou assíncronos
	  	\item<3-> Infromação abstraída para tipo de dados comum
		\item<4-> Utiliza a API para descobrir o respectivo manager
        \end{itemize}
 \end{columns}
\end{frame}
\note{Speak clearly}  % Add notes to yourself that will be displayed when
                      % typeset with the notes or notesonly class options


\section{Implementação}

\begin{frame}
  \MyLogo
  \frametitle{Ambiente}\framesubtitle{}   % Insert frame title between curly braces
\begin{columns}
   \column{.5\textwidth}
	\begin{itemize}\setlength{\itemsep}{6mm}
               %\item<1-> Exemplo de topologia física de rede  %(Click ``Next Page'' to see Point 2) % Use Next Page to go to Point 2
	  	\item<1-> Boards TS-7500
	  	\item<2-> Processador ARM 250MHz
		  \item<3-> 64MB de RAM 
	  	\item<4-> Debian Linux
	  	\item<5-> Linguagem C
        \end{itemize}
   \column{.5\textwidth}
       \pgfuseimage{ts7500}        
 \end{columns}
\end{frame}
\note{The end}       % Add notes to yourself that will be displayed when
		     % typeset with the notes or notesonly class options

\begin{frame}
  \MyLogo
  \frametitle{Mapas}\framesubtitle{}   % Insert frame title between curly braces
\begin{columns}
   \column{.7\textwidth}
	\begin{itemize}\setlength{\itemsep}{6mm}
               %\item<1-> Exemplo de topologia física de rede  %(Click ``Next Page'' to see Point 2) % Use Next Page to go to Point 2
	  	\item<1-> Representação por células hexagonais
	  	\item<2-> Células de dimensão variável
	  	\item<3-> Células de transição para outros mapas
        \end{itemize}
   \column{.3\textwidth}
       \pgfuseimage{cells11-5}        
 \end{columns}
\end{frame}


\subsection{Protótipo IST-Taguspark}

\begin{frame}
  \MyLogo
  \frametitle{Protótipo}\framesubtitle{Plugins}   % Insert frame title between curly braces

  \begin{itemize}
  \item<1-> Feixes de infra-vermelhos para detectar entradas e saídas de pessoas
	\begin{itemize}
  		\item<2-> Curto alcance que é simulateamente emissor e receptor
		\item<2-> Par emissor e receptor infra-vermelho de longo alcance
		\item<3-> Uso de dois feixes em paralelo para detectar direcção
%		\item<4-> Dispositivos de longo alcance usados de forma síncrona para não causar interferências
	\end{itemize}
  \item<5-> Tecnologia Bluetooth para detectar a força de sinal de aparelhos pessoais 
	\begin{itemize}
  		\item<6-> Transformar força de sinal em distância ao nó
		\item<7-> Através de leis de propagação em espaço livre
		\item<7-> Através de fingerprinting
	\end{itemize}
  \end{itemize}
\end{frame}
\note{Speak clearly}  % Add notes to yourself that will be displayed when
                      % typeset with the notes or notesonly class options


\begin{frame}
  \MyLogo
  \frametitle{Protótipo}\framesubtitle{Hardware}   % Insert frame title between curly braces
	
   \begin{columns}
   \column{.5\textwidth}
	\begin{center}
        \pgfuseimage{ir_proto}
	\end{center}
   \column{.5\textwidth}
	\begin{center}
        \pgfuseimage{ir_emitters}
	\end{center}
 \end{columns}
   \begin{columns}
   \column{.5\textwidth}
	\begin{center}
	\pgfuseimage{bt_proto}
	\end{center}
   \column{.5\textwidth}
	\begin{center}
	\pgfuseimage{ir_receivers}
	\end{center}
 \end{columns}
	
\end{frame}
\note{Speak clearly}  % Add notes to yourself that will be displayed when
                      % typeset with the notes or notesonly class options

\section{Resultados}
\begin{frame}
  \MyLogo
  \frametitle{Resultados}\framesubtitle{Ambiente de Testes}   % Insert frame title between curly braces
	\begin{columns}
   \column{.5\textwidth}
	\begin{itemize}\setlength{\itemsep}{4mm}
               %\item<1-> Exemplo de topologia física de rede  %(Click ``Next Page'' to see Point 2) % Use Next Page to go to Point 2
	  	\item<1-> Duas zonas de contagem de pessoas
	  	\item<2-> Uma zona de localização de pessoas
	  	\item<3-> Nós A e D detectam entradas e saídas
		\item<4-> Nós A, B e C realizam a triangulação
        \end{itemize}
   \column{.5\textwidth}
       \pgfuseimage{planta11}        
 \end{columns}
\end{frame}
\note{The end}       % Add notes to yourself that will be displayed when
		     % typeset with the notes or notesonly class options

\subsection{Contagem de pessoas}

\begin{frame}
  \MyLogo
  \frametitle{Resultados}\framesubtitle{Funcionamento do Sensor Infra-Vermelhos}   % Insert frame title between curly braces

\tiny
\begin{center}
\begin{tabular}{ | l | p{7cm} |}

\hline
	\textbf{Test} & \textbf{Result} \\ \hline
	Entrada/saída normal & Uma entrada/saída \\ \hline
	Entrada/saída rápida (a correr) & Se o sensor estiver obstruído por um intervalo inferior a 10ms poderá não detectar a passagem \\ \hline
	Passagem cruzada & Usualmente detecta apenas uma das direcções \\ \hline
	Passagem simultânea & Se os feixes não deixarem de estar obstruídos durante as duas passagens, apenas uma é detectada\\
\hline

\end{tabular} 
\end{center}
\normalsize
	\begin{itemize}\setlength{\itemsep}{2mm}
	  	\item<2-> Limitação de leitura a uma frequência de 100Hz
	  	\item<3-> Mais indicado para passagens singulares
        \end{itemize}
\end{frame}
\note{Speak clearly}  % Add notes to yourself that will be displayed when
                      % typeset with the notes or notesonly class options

\begin{frame}
  \MyLogo
  \frametitle{Resultados}\framesubtitle{Análise ao longo do dia}   % Insert frame title between curly braces
   \begin{columns}
   \column{.56\textwidth}
	\begin{itemize}\setlength{\itemsep}{5mm}
		\item É possível verificar a variação do número de pessoas na àrea ao longo do dia
		\item A área exterior teve um máximo de 18 pessoas
		\item A área interior teve um máximo de 5 pessoas
	\end{itemize}
   \column{.44\textwidth}
	\begin{columns}
		\column{1\textwidth}
		\pgfuseimage{test1.2}
	\end{columns}
	\begin{columns}
		\column{1\textwidth}
		\pgfuseimage{test1.3}
	\end{columns}
   \end{columns}
\end{frame}
\note{Speak clearly}  % Add notes to yourself that will be displayed when
                      % typeset with the notes or notesonly class options
\begin{frame}
  \MyLogo
  \frametitle{Resultados}\framesubtitle{Análise ao longo do dia} 
   \begin{columns}
   \column{.56\textwidth}
	\begin{itemize}\setlength{\itemsep}{5mm}
		\item É possível verificar a variação do número de pessoas na àrea ao longo do dia
		\item A área exterior teve um máximo de 18 pessoas
		\item A área interior teve um máximo de 5 pessoas
	\end{itemize}
   \column{.44\textwidth}
	\begin{columns}
		\column{1\textwidth}
		\pgfuseimage{test2.2}
	\end{columns}
	\begin{columns}
		\column{1\textwidth}
		\pgfuseimage{test2.3}
	\end{columns}
   \end{columns}
\end{frame}

\begin{frame}
  \MyLogo
  \frametitle{Resultados}\framesubtitle{Análise ao longo do dia} 
   \begin{columns}
   \column{.56\textwidth}
	\begin{itemize}\setlength{\itemsep}{5mm}
		\item É possível verificar a variação do número de pessoas na àrea ao longo do dia
		\item A área exterior teve um máximo de 18 pessoas
		\item A área interior teve um máximo de 5 pessoas
	\end{itemize}
   \column{.44\textwidth}
	\begin{columns}
		\column{1\textwidth}
		\pgfuseimage{test3.2}
	\end{columns}
	\begin{columns}
		\column{1\textwidth}
		\pgfuseimage{test3.3}
	\end{columns}
   \end{columns}
\end{frame}

\subsection{Rastreamento de pessoas}

\begin{frame}
  \MyLogo
  \frametitle{Resultados}\framesubtitle{Rastreamento de pessoas}   % Insert frame title between curly braces
	\begin{columns}
   \column{.5\textwidth}
	\begin{itemize}\setlength{\itemsep}{4mm}
               %\item<1-> Exemplo de topologia física de rede  %(Click ``Next Page'' to see Point 2) % Use Next Page to go to Point 2
	  	\item<1-> Movimento real vs. detectado
		\item<2-> Movimento detectado não é contínuo devido à perda de sinal
		\item<3-> Pontos adjacentes correspondem na realidade à mesma célula
        \end{itemize}
   \column{.5\textwidth}
       \pgfuseimage{test_mov}        
 \end{columns}
\end{frame}
\note{Speak clearly}  % Add notes to yourself that will be displayed when
                      % typeset with the notes or notesonly class options

\section{Conclusões}

\begin{frame}
  \MyLogo
  \frametitle{Conclusões}   % Insert frame title between curly braces
  \begin{itemize}\setlength{\itemsep}{6mm}
	  \item<1-> Criação de plataforma de localização escalável
	  \item<2-> Capaz de integrar variadas tecnologias de comunicação a custo reduzido
	  \item<3-> Outros projectos podem dar uso a esta plataforma para aplicar políticas de poupança energética
	  \item<4-> Criação de um add-on de descoberta de serviços a plataforma multi-funcional de monitorização
  \end{itemize}
\end{frame}
\note{The end}       % Add notes to yourself that will be displayed when
		     % typeset with the notes or notesonly class options

%\subsection{Trabalho Futuro}

\begin{frame}
  \MyLogo
  \frametitle{Conclusões}\framesubtitle{Trabalho Futuro}   % Insert frame title between curly braces

  \begin{itemize}\setlength{\itemsep}{7mm}
	  \item<1-> Carregar mapas dinâmicamente à medida que são necessários
	  \item<2-> Realizar aproximações das distâncias para obter sempre triangulações solucionáveis
	  \item<3-> Providenciar mecanismos de reset para quando é detectada adulteração de dados
  \end{itemize}
\end{frame}
\note{Speak clearly}  % Add notes to yourself that will be displayed when
                      % typeset with the notes or notesonly class options

\begin{frame}{Fim da apresentação}
 \MyLogo
 \begin{center}
  \Large Obrigado pela vossa comparência.\\
  \vspace{0.5cm}
  \large \bf Questões?
 \end{center}
\end{frame}

\end{document}
