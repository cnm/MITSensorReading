\documentclass[]{beamer}
% Class options include: notes, notesonly, handout, trans,
%                        hidesubsections, shadesubsections,
%                        inrow, blue, red, grey, brown
\mode<presentation>
{
  \usetheme{CambridgeUS}
  % or ...
  \usecolortheme{beaver}
  \setbeamercovered{transparent}
  % or whatever (possibly just delete it)
}
\setbeamertemplate{navigation symbols}{}


\usepackage[english]{babel}
% or whatever

\usepackage[utf8]{inputenc}
% or whatever

\usepackage{times}
\usepackage[T1]{fontenc}
% Or whatever. Note that the encoding and the font should match. If T1
% does not look nice, try deleting the line with the fontenc.
\usepackage[absolute,overlay]{textpos}


\title[People Counting and Patterns of Movement] % (optional, use only with long paper titles)
{Green IT - People Counting and Detection of Patterns of Movement}

\subtitle
{}
\author[António Calçada Novais] % (optional, use only with lots of authors)
{António Calçada Novais \\ Student nº 57432 \\ MERC}
% - Give the names in the same order as the appear in the paper.
% - Use the \inst{?} command only if the authors have different
%   affiliation.

\institute[IST] % (optional, but mostly needed)
{
  INESC-ID, Communications Networks and Mobility\\
  Instituto Superior Técnico - Campus TagusPark
}
% - Use the \inst command only if there are several affiliations.
% - Keep it simple, no one is interested in your street address.

\date % (optional, should be abbreviation of conference name)
{November 11, 2011}
% - Either use conference name or its abbreviation.
% - Not really informative to the audience, more for people (including
%   yourself) who are reading the slides online

\subject{Computer Science}
% This is only inserted into the PDF information catalog. Can be left
% out. 



% If you have a file called "university-logo-filename.xxx", where xxx
\pgfdeclareimage[height=0.6cm]{logoIST}{figures/logoIST.png}
\pgfdeclareimage[height=0.8cm]{logoINESC}{figures/inescID.png}
\pgfdeclareimage[height=0.3cm]{logoMIT}{figures/mit-logo.png}
% is a graphic format that can be processed by latex or pdflatex,
% resp., then you can add a logo as follows:


\setlength{\TPHorizModule}{1mm}
\setlength{\TPVertModule}{1mm}
\newcommand{\MyLogo}{%
\begin{textblock}{14}(105.0,5.0)
  \vbox{\hbox to 2.17cm{\hfil\pgfuseimage{logoIST}}\vskip-0.8cm\hbox{\pgfuseimage{logoINESC}}\vskip-0.6cm\hskip-1.4cm\hbox{\pgfuseimage{logoMIT}}}
\end{textblock}
}

%\logo{}

\AtBeginSection[] {
	\begin{frame}<beamer>{Índice}
	\MyLogo
	\tableofcontents[currentsection,subsubsectionstyle=hide]
	\end{frame}
}



% Theme for beamer presentation.
%\usepackage{beamerthemesplit} 
% Other themes include: beamerthemebars, beamerthemelined, 
%                       beamerthemetree, beamerthemetreebars  

%\title{Green IT - People Counting and Detection of Patterns of Movement}    % Enter your title between curly braces
%\author{António Calçada Novais}                 % Enter your name between curly braces
%\institute{NESC-ID, Communications Networks and Mobility\\Instituto Superior Técnico - Taguspark}      % Enter your institute name between curly braces
%\date{\today}                    % Enter the date or \today between curly braces

\begin{document}

% Creates title page of slide show using above information
\begin{frame}
  \MyLogo
  \titlepage
\end{frame}
\note{Talk for 20 minutes} % Add notes to yourself that will be displayed when
                           % typeset with the notes or notesonly class options

\section[Índice]{}

% Creates table of contents slide incorporating
% all \section and \subsection commands
\begin{frame}
  \MyLogo
  \frametitle{Índice}
  \tableofcontents
\end{frame}

\section{Introdução}

\subsection{Motivação}

\begin{frame}
  \MyLogo 
  \frametitle{Motivação}   % Insert frame title between curly braces

  \begin{itemize}
	\item Dificuldade de gerir consumos energéticos num campus
	\item Redução de custos energéticos é uma necessidade
	\item Previsível diminuição dos recursos naturais
	\item Disseminação da computação pervasiva 
  \end{itemize}
\end{frame}
\note[enumerate]       % Add notes to yourself that will be displayed when
{                      % typeset with the notes or notesonly class options
\item Note for Point 1   
\item Note for Point 2   
}

\subsection{Objectivos}

\begin{frame}
  \frametitle{Objectivos}   % Insert frame title between curly braces

  \begin{itemize}
	\item Desenhar um sistema capaz de contar pessoas e identificar os seus padrões de mobilidade dentro de edifícios
	\item O custo do sistema deve ser inferior às poupanças
	\item Caso de uso de plataforma multi-funcional de monitorização
	\item Integrar na plataforma supra-mencionada módulo de descoberta de serviços
  \end{itemize}
\end{frame}
\note[enumerate]       % Add notes to yourself that will be displayed when
{                      % typeset with the notes or notesonly class options
\item Note for Point 1   
\item Note for Point 2   
}

\begin{frame}
  \frametitle{Vantagens de utilizar a plataforma multi-functional}   % Insert frame title between curly braces

  \begin{itemize}
	\item Abstracção de problemas de comunicação
	\begin{itemize}
		\item Sincronização
		\item Estado da rede
		\item Descoberta de Serviços
	\end{itemize}

	\item Optimizar o processamento à capacidade do nó
  \end{itemize}
\end{frame}
\note[enumerate]       % Add notes to yourself that will be displayed when
{                      % typeset with the notes or notesonly class options
\item Note for Point 1
\item Note for Point 2
}

%\section{State of the art}

%\begin{frame}
%  \frametitle{State of the art}   % Insert frame title between curly braces

%  \begin{itemize}
%  \item Point 1
%  \item Point 2
%  \item Point 3
%  \end{itemize}
%\end{frame}
%\note[enumerate]       % Add notes to yourself that will be displayed when
%{                      % typeset with the notes or notesonly class options
%\item Note for Point 1   
%\item Note for Point 2   
%}%

%\subsection{Service Discovery}

%\begin{frame}
%  \frametitle{Simple slide with three points shown in succession}   % Insert frame title between curly braces%
%
 % \begin{itemize}
  %\item<1-> Point 1 (Click ``Next Page'' to see Point 2) % Use Next Page to go to Point 2
  %\item<2-> Point 2  % Use Next Page to go to Point 3
  %\item<3-> Point 3
  %\end{itemize}
%\end{frame}
%\note{Speak clearly}  % Add notes to yourself that will be displayed when
                      % typeset with the notes or notesonly class options

%\subsection{Location Systems}

%\begin{frame}
%  \frametitle{Simple slide with three points shown in succession}   % Insert frame title between curly braces%

%  \begin{itemize}
%  \item<1-> Point 1 (Click ``Next Page'' to see Point 2) % Use Next Page to go to Point 2
 % \item<2-> Point 2  % Use Next Page to go to Point 3
  %\item<3-> Point 3
  %\end{itemize}
%\end{frame}
%\note{Speak clearly}  % Add notes to yourself that will be displayed when
                      % typeset with the notes or notesonly class options



\section{Arquitectura}


\subsection{Sistema de localização}

\begin{frame}
  \frametitle{Requisitos}   % Insert frame title between curly braces

  \begin{itemize}
  \item<1-> Point 1 (Click ``Next Page'' to see Point 2) % Use Next Page to go to Point 2
  \item<3-> Point 3
  \item<2-> Point 2  % Use Next Page to go to Point 3
  \end{itemize}
\end{frame}
\note{Speak clearly}  % Add notes to yourself that will be displayed when
                      % typeset with the notes or notesonly class options

\begin{frame}
  \frametitle{Topologia}   % Insert frame title between curly braces

  \begin{itemize}
  \item<1-> Point 1 (Click ``Next Page'' to see Point 2) % Use Next Page to go to Point 2
  \item<2-> Point 2  % Use Next Page to go to Point 3
  \item<3-> Point 3
  \end{itemize}
\end{frame}
\note{Speak clearly}  % Add notes to yourself that will be displayed when
                      % typeset with the notes or notesonly class options

\subsection{Plataforma Multi-Funcional de Monitorização}

\begin{frame}
  \frametitle{Overview}   % Insert frame title between curly braces

  \begin{itemize}
  \item<1-> Point 1 (Click ``Next Page'' to see Point 2) % Use Next Page to go to Point 2
  \item<3-> Point 3
  \item<2-> Point 2  % Use Next Page to go to Point 3
  \end{itemize}
\end{frame}
\note{Speak clearly}  % Add notes to yourself that will be displayed when
                      % typeset with the notes or notesonly class options

\begin{frame}
  \frametitle{Descoberta de Serviços}   % Insert frame title between curly braces

  \begin{itemize}
  \item<1-> Point 1 (Click ``Next Page'' to see Point 2) % Use Next Page to go to Point 2
  \item<2-> Point 2  % Use Next Page to go to Point 3
  \item<3-> Point 3
  \end{itemize}
\end{frame}
\note{Speak clearly}  % Add notes to yourself that will be displayed when
                      % typeset with the notes or notesonly class options

\subsection{Componentes de localização}

\begin{frame}
  \frametitle{Spotter}   % Insert frame title between curly braces

  \begin{itemize}
  \item<1-> Point 1 (Click ``Next Page'' to see Point 2) % Use Next Page to go to Point 2
  \item<3-> Point 3
  \item<2-> Point 2  % Use Next Page to go to Point 3
  \end{itemize}
\end{frame}
\note{Speak clearly}  % Add notes to yourself that will be displayed when
                      % typeset with the notes or notesonly class options


\section{Implementação}

\begin{frame}
  \frametitle{Ambiente}   % Insert frame title between curly braces
  \begin{columns}[c]
  \column{2in}  % slides are 3in high by 5in wide
  \begin{itemize}
  \item<1-> First item
  \item<2-> Second item
  \item<3-> ...
  \end{itemize}
  \column{2in}
  \framebox{Insert graphic here % e.g. \includegraphics[height=2.65in]{graphic}
  }
  \end{columns}
\end{frame}
\note{The end}       % Add notes to yourself that will be displayed when
		     % typeset with the notes or notesonly class options


\subsection{Protótipo IST-Taguspark}

\begin{frame}
  \frametitle{Simple slide with three points shown in succession}   % Insert frame title between curly braces

  \begin{itemize}
  \item<1-> Point 1 (Click ``Next Page'' to see Point 2) % Use Next Page to go to Point 2
  \item<2-> Point 2  % Use Next Page to go to Point 3
  \item<3-> Point 3
  \end{itemize}
\end{frame}
\note{Speak clearly}  % Add notes to yourself that will be displayed when
                      % typeset with the notes or notesonly class options

\subsection{Plugins}

\begin{frame}
  \frametitle{Simple slide with three points shown in succession}   % Insert frame title between curly braces

  \begin{itemize}
  \item<1-> Point 1 (Click ``Next Page'' to see Point 2) % Use Next Page to go to Point 2
  \item<2-> Point 2  % Use Next Page to go to Point 3
  \item<3-> Point 3
  \end{itemize}
\end{frame}
\note{Speak clearly}  % Add notes to yourself that will be displayed when
                      % typeset with the notes or notesonly class options

\section{Resultados}

\begin{frame}
  \frametitle{Ambiente de Testes}   % Insert frame title between curly braces
  \begin{columns}[c]
  \column{2in}  % slides are 3in high by 5in wide
  \begin{itemize}
  \item<1-> First item
  \item<2-> Second item
  \item<3-> ...
  \end{itemize}
  \column{2in}
  \framebox{Insert graphic here % e.g. \includegraphics[height=2.65in]{graphic}
  }
  \end{columns}
\end{frame}
\note{The end}       % Add notes to yourself that will be displayed when
		     % typeset with the notes or notesonly class options

\subsection{Contagem de pessoas}

\begin{frame}
  \frametitle{Funcionamento do Sensor Infra-Vermelhos}   % Insert frame title between curly braces

  \begin{itemize}
  \item<1-> Point 1 (Click ``Next Page'' to see Point 2) % Use Next Page to go to Point 2
  \item<2-> Point 2  % Use Next Page to go to Point 3
  \item<3-> Point 3
  \end{itemize}
\end{frame}
\note{Speak clearly}  % Add notes to yourself that will be displayed when
                      % typeset with the notes or notesonly class options

\begin{frame}
  \frametitle{Análise ao longo do dia}   % Insert frame title between curly braces

  \begin{itemize}
  \item<1-> Point 1 (Click ``Next Page'' to see Point 2) % Use Next Page to go to Point 2
  \item<2-> Point 2  % Use Next Page to go to Point 3
  \item<3-> Point 3
  \end{itemize}
\end{frame}
\note{Speak clearly}  % Add notes to yourself that will be displayed when
                      % typeset with the notes or notesonly class options

\subsection{Rastreamento de pessoas}

\begin{frame}
  \frametitle{Experimentação}   % Insert frame title between curly braces

  \begin{itemize}
  \item<1-> Point 1 (Click ``Next Page'' to see Point 2) % Use Next Page to go to Point 2
  \item<2-> Point 2  % Use Next Page to go to Point 3
  \item<3-> Point 3
  \end{itemize}
\end{frame}
\note{Speak clearly}  % Add notes to yourself that will be displayed when
                      % typeset with the notes or notesonly class options

\section{Conclusões}

\begin{frame}
  \frametitle{Conclusões}   % Insert frame title between curly braces
  \begin{columns}[c]
  \column{2in}  % slides are 3in high by 5in wide
  \begin{itemize}
  \item<1-> First item
  \item<2-> Second item
  \item<3-> ...
  \end{itemize}
  \column{2in}
  \framebox{Insert graphic here % e.g. \includegraphics[height=2.65in]{graphic}
  }
  \end{columns}
\end{frame}
\note{The end}       % Add notes to yourself that will be displayed when
		     % typeset with the notes or notesonly class options

\subsection{Trabalho Futuro}

\begin{frame}
  \frametitle{Trabalho Futuro}   % Insert frame title between curly braces

  \begin{itemize}
  \item<1-> Point 1 (Click ``Next Page'' to see Point 2) % Use Next Page to go to Point 2
  \item<2-> Point 2  % Use Next Page to go to Point 3
  \item<3-> Point 3
  \end{itemize}
\end{frame}
\note{Speak clearly}  % Add notes to yourself that will be displayed when
                      % typeset with the notes or notesonly class options

\end{document}
